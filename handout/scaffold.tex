% Define the module top matter
% This gets used to create the chapter title page
% NOTES:
%  * When multiple people have authored or contributed to the module, simply use a LaTeX line break
%    (a double-backslash: \\) at the end of each person.
%  * If you don't want this information shown on the module chapter page, simply remove the lines
%    within the \setModuleAuthors{} and \setModuleContributions{} environments
\setModuleTitle{Scaffolding}

\setModuleAuthors{%
  Bernardo Clavijo, TGAC \mailto{Bernardo.Clavijo@tgac.ac.uk} \\
}
\setModuleContributions{%
  Sonika Tyagi, AGRF \mailto{sonika.tyagi@agrf.org.au} \\
  Zhiliang Chan \mailto{zhiliang@unsw.edu.au}
}

\chapter{\moduleTitle}
\newpage

\section{Key Learning Outcomes}

After completing this module the trainee should be able to:
\begin{itemize}
  \item Run an assembly using long mate pair (LMP) reads. 
  \item Compare results of LMP assemblies with those from the short reads.
  \item Run scaffolding and assess the assembly statistics`

\end{itemize}
% END KLOs

\section{Resources You'll be Using}
 
\subsection{Tools Used}
\begin{description}[style=multiline,labelindent=0cm,align=left,leftmargin=0.5cm]
  \item[Kmer Analysis Tool kit]\hfill\\
  	\url{https://github.com/TGAC/KAT}
  \item[Nextclip]\hfill\\
  	\url{https://github.com/richardmleggett/nextclip}
  \item[Abyss]\hfill\\
  	\url{http://www.bcgsc.ca/platform/bioinfo/software/abyss}
  \item[Soap Denovo]\hfill\\
  	\url{http://soap.genomics.org.cn/soapdenovo.html}
  \item[SOAPec]\hfill\\
  	\url{http://soap.genomics.org.cn/about.html}
\end{description}



\newpage

% BEGIN: Introduction
\section{Scaffolding with long mate pair libraries: Chalara scaffolding using LMP}

\begin{information}
Scaffold is made of two or more contigs joined together using the read pair information. In the absence of a high-quality reference genome, new genome assemblies are often evaluated on the basis of the number of scaffolds and contigs required to represent the genome. Other criteria such as the alignment of reads back to assemblies, N50,  and the length of contigs and scaffolds relative to the size of the genome can also be used to assess the quality of assemblies.
In this excercise we will be using a long mate pair data to perform assembly and scaffolding and we will focus on how using LMP reads can affect the assemblies.
Most part of this tutorial has been precomputed and made available to you in interest of time. We will spend more time exploring and discussing the results.
\end{information}
\subsection{Pair-end assembly using Chalara dataset}
Before doing a scaffolding, we need to build an assembly using the pair-end reads first.


%%%% need to confirm the pre-computed bits with BJ 
\begin{warning}
You DO NOT need to run this command. This has already been done for you.
\begin{lstlisting}

abyss-pe name=cha1 k=27 in="../LIB2570_raw_R1.fastq ../LIB2570_raw_R2.fastq" np=4

\end{lstlisting}
\end{warning}
\beggin{steps}
Let's look at the stats by doing:
\begin{lstlisting}
less cha1-stats.tab
\end{lstlisting}
\end{steps}


\begin{table}[H]
  \centering
  \caption{Statistics of \textit{Chalara} assembly by ABySS using k=27}
    \begin{tabular}{rrrrrrrrrrr}
    \toprule
    \textbf{n} & \textbf{n:500} & \textbf{L50} & \textbf{min} & \textbf{N80}& \textbf{N50}& \textbf{N20}& \textbf{E-size}& \textbf{max} & \textbf{sum}& \textbf{name}\\
    \midrule
    394789  & 11681  & 2094  & 500  & 2880  & 6254  & 12303  & 7936    & 44414  & 44.07e6  & cha1-unitigs.fa
	394199  & 11673  & 2097  & 500  & 2887  & 6255  & 12303  & 7937    & 44414  & 44.11e6  & cha1-contigs.fa
	394161  & 11647  & 2095  & 500  & 2898  & 6269  & 12303  & 7944    & 44414  & 44.12e6  & cha1-scaffolds.fa
    \bottomrule
    \end{tabular}
  \label{tab:chak27}
\end{table}


\begin{warning}
This is also a pre-computed example for you:
\begin{lstlisting}

abyss-pe name=cha2 k=61 in="../LIB2570_raw_R1.fastq ../LIB2570_raw_R2.fastq" np=4
\end{lstlisting}
\end{warning}

\begin{steps}
Again, we need to look at the statstics of our new assembly using k=61:
\begin{lstlisting}
less cha2-stats.tab
\end{lstlisting}
\end{steps}

It should be looking like this:
\begin{table}[H]
  \centering
  \caption{Statistics of \textit{Chalara} assembly by ABySS using k=61}
    \begin{tabular}{rrrrrrrrrrr}
    \toprule
    \textbf{n} & \textbf{n:500} & \textbf{L50} & \textbf{min} & \textbf{N80}& \textbf{N50}& \textbf{N20}& \textbf{E-size}& \textbf{max} & \textbf{sum}& \textbf{name}\\
    \midrule
	130547  & 12596  & 1770  & 500  & 3352  & 8379  & 16380  & 10518   & 54300  & 50.75e6  & cha2-unitigs.fa
	130210  & 12575  & 1771  & 500  & 3363  & 8382  & 16380  & 10525   & 54300  & 50.78e6  & cha2-contigs.fa
	130182  & 12555  & 1769  & 500  & 3377  & 8394  & 16380  & 10534   & 54300  & 50.78e6  & cha2-scaffolds.fa
    \bottomrule
    \end{tabular}
  \label{tab:chak61}
\end{table}

\begin{questions}
The assembly contiguity is worse than fusarium, why?
\begin{answer}
\item Genome characteristics.
\item Data:
\item Coverage
\item Error distributions
\item Read sizes
\item Fragment sizes
\item  Kmer spectra
\end{answer}
\end{questions}

\subsection{Chalara scaffolding using long mate-pair reads(LMP)}

Let's put some LMP in there.

\begin{steps}
\begin{lstlisting}
abyss-pe name=chalmp1 k=61 in="../LIB2570_raw_R1.fastq ../LIB2570_raw_R2.fastq" mp="lmp1" lmp1="../LIB8209_raw_R1.fastq ../LIB8209_raw_R2.fastq" np=4
less chalmp1-stats.tab
\end{lstlisting}
\end{steps}

\begin{table}[H]
  \centering
  \caption{Statistics of \textit{Chalara} assembly by ABySS using k=27}
    \begin{tabular}{rrrrrrrrrrr}
    \toprule
    \textbf{n} & \textbf{n:500} & \textbf{L50} & \textbf{min} & \textbf{N80}& \textbf{N50}& \textbf{N20}& \textbf{E-size}& \textbf{max} & \textbf{sum}& \textbf{name}\\
    \midrule
	130548  & 12596  & 1770  & 500  & 3352  & 8379  & 16380  & 10518   & 54300  & 50.75e6  & chalmp1-unitigs.fa
	130211  & 12575  & 1771  & 500  & 3363  & 8382  & 16380  & 10525   & 54300  & 50.78e6  & chalmp1-contigs.fa
	130148  & 12545  & 1769  & 500  & 3380  & 8400  & 16380  & 10535   & 54300  & 50.78e6  & chalmp1-scaffolds.fa
    \bottomrule
    \end{tabular}
  \label{tab:chak27}
\end{table}

\begin{information}
So... what happened?
Let's check the data by:
\item Kmer spectra
\item Fragment sizes
\item Any hints on the protocol?
A not-so-obvious property:
\end{information}

\begin{steps}
\begin{lstlisting}
kat plot spectra-mx --intersection -x 30 -y 15000000 -o pe_vs_lmp-main.mx.spectra-mx.png pe_vs_lmp-main.mx

\end{lstlisting}
\end{steps}

\begin{information}
Do you remember LMP require pre-processing?  Let's try with processed LMP.
\end{information}

\begin{steps}
Prior task (already made) preprocess the LMP with nextclip.

\begin{lstlisting}
abyss-pe name=chalmpproc1 k=61 in="../../cha_raw/LIB2570_raw_R1.fastq ../../cha_raw/LIB2570_raw_R2.fastq" mp="proclmp1" proclmp1="../LIB8209_preproc_R1.fastq ../LIB8209_preproc_R2.fastq" np=4
less chalmpproc1-stats.tab
\end{lstlisting}
\end{steps}

\begin{table}[H]
  \centering
  \caption{Statistics of \textit{Chalara} assembly by ABySS using k=61 with LMP}
    \begin{tabular}{rrrrrrrrrrr}
    \toprule
    \textbf{n} & \textbf{n:500} & \textbf{L50} & \textbf{min} & \textbf{N80}& \textbf{N50}& \textbf{N20}& \textbf{E-size}& \textbf{max} & \textbf{sum}& \textbf{name}\\
    \midrule
	130548  & 12596  & 1770  & 500  & 3352   & 8379   & 16380   & 10518   & 54300   & 50.75e6  & chalmpproc1-unitigs.fa
	130211  & 12575  & 1771  & 500  & 3363   & 8382   & 16380   & 10525   & 54300   & 50.78e6  & chalmpproc1-contigs.fa
	122061  & 6679   & 167   & 500  & 18609  & 87510  & 187171  & 106178  & 397967  & 51.15e6  & chalmpproc1-scaffolds.fa
    \bottomrule
    \end{tabular}
  \label{tab:chaklmpk61}
\end{table}

That's much better!

\section{Chalara: beyond first pass}
Do you remember these?

Genome characteristics.
Data:
 \item Coverage
 \item Error distributions
 \item Read sizes
 \item Fragment sizes
 \item Kmer spectra

Let's think a bit and try to improve the assembly...

\begin{questions}
If you look at the pre-processed LMP, do you notice anything peculiar?
\begin{answer}
Needs to fill
\end{answer}
\end{questions}
\begin{steps}
Testing the inclussion of heavily pre-procesed LMP coverage into the DBG
\begin{lstlisting}
abyss-pe name=chalmp2 k=61 se="../LIB8209_preproc_single.fastq" lib="lmp1" lmp1="../LIB8209_preproc_R1.fastq ../LIB8209_preproc_R2.fastq" np=4 >chaproclmp2.log 2>&1
less chalmp2-stats.tab
\end{lstlisting}
\end{steps}
Let's look at the stats.
\begin{table}[H]
  \centering
  \caption{Statistics of improved \textit{Chalara} assembly by ABySS using k=61 with LMP}
    \begin{tabular}{rrrrrrrrrrr}
    \toprule
    \textbf{n} & \textbf{n:500} & \textbf{L50} & \textbf{min} & \textbf{N80}& \textbf{N50}& \textbf{N20}& \textbf{E-size}& \textbf{max} & \textbf{sum}& \textbf{name}\\
    \midrule
	128306  & 10150  & 1182  & 500  & 4954   & 12585   & 24777   & 15693   & 68684   & 51.03e6  & chaproclmp4-unitigs.fa
	118939  & 5772   & 362   & 500  & 15717  & 41870   & 82033   & 52819   & 252066  & 52.24e6  & chaproclmp4-contigs.fa
	116139  & 4014   & 141   & 500  & 41145  & 109273  & 224986  & 133450  & 460979  & 52.56e6  & chaproclmp4-scaffolds.fa
    \bottomrule
    \end{tabular}
  \label{tab:chaklmp2-k61}
\end{table}

\section{References}

%TODO Change to using BiBTeX
\begin{enumerate}
  \item Robert Ekblom* andJochen B. W. Wolf. "A field guide to whole-genome sequencing, assembly and annotation" Evolutionary Applications Special Issue: Evolutionary Conservation Volume 7, Issue 9, pages 1026–1042, November 2014
  \item De novo genome assembly: what every biologist should know Nature Methods 9, 333–337 (2012) doi:10.1038/nmeth.1935 
\end{enumerate}
